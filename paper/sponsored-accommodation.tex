\documentclass[12pt,letterpaper]{article}
\usepackage[latin1]{inputenc}
\usepackage{amsmath}
\usepackage{amsfonts}
\usepackage{amssymb}
\usepackage{graphicx}
\usepackage{framed}
\usepackage{pdfpages}
\usepackage{eso-pic}

\pagenumbering{gobble}

\author{Jonas Kersulis}
\title{Request for Sponsored Accommodation}


\AddToShipoutPictureBG{%
  \AtPageUpperLeft{\raisebox{-\height}{\includegraphics[trim=-1in -1in -1in -1in,clip, width=4.5in]{../images/ECE-marketing-formal}}}%
}

\begin{document}


\noindent Dear IEEE and PowerTech Local Organizing Committee:
\vspace{20pt}

My name is Jonas Kersulis, and I am a male PhD student at the University of Michigan in Ann Arbor, MI, United States. I am in the second year of my program, and this is my first time publishing. PowerTech is one of the largest conferences in my field, and I would love to experience it.

\ 

The title of my paper is ``Temperature-based Instanton Analysis: Identifying Vulnerability in Transmission Networks'' (abstract ID number 863). In this paper my co-authors and I analyze the effects of wind variation on transmission networks. The flexibility of instanton analysis makes it a powerful, adaptable tool for addressing wind variation in the midst of today's network diversity. Wind energy is handled differently across regions and nations, and a key part of our work is to solicit feedback from as many system operators and fellow grid practitioners as we can. In fact, the idea for temporal instanton analysis (the subject of my PowerTech paper) came from a serendipitous encounter with a Texas system operator in Washington, D.C. Interactions like this one have proven invaluable, and I have no doubt there will be many such opportunities at Eindhoven this summer.

\ 

Thank you for organizing PowerTech, and for the opportunity to request sponsored accommodation.

\vspace{100pt}

\noindent Best Regards,
\begin{figure}[h!]
\includegraphics[trim=0in 0in 0in 0in, clip, width=0.3\linewidth]{../images/signature}
\label{fig:signature}
\end{figure}

Jonas A. Kersulis


\end{document}