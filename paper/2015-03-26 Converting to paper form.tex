
% Default to the notebook output style

    


% Inherit from the specified cell style.




    
\documentclass{article}

    
    
    \usepackage{graphicx} % Used to insert images
    \usepackage{adjustbox} % Used to constrain images to a maximum size 
    \usepackage{color} % Allow colors to be defined
    \usepackage{enumerate} % Needed for markdown enumerations to work
    \usepackage{geometry} % Used to adjust the document margins
    \usepackage{amsmath} % Equations
    \usepackage{amssymb} % Equations
    \usepackage{eurosym} % defines \euro
    \usepackage[mathletters]{ucs} % Extended unicode (utf-8) support
    \usepackage[utf8x]{inputenc} % Allow utf-8 characters in the tex document
    \usepackage{fancyvrb} % verbatim replacement that allows latex
    \usepackage{grffile} % extends the file name processing of package graphics 
                         % to support a larger range 
    % The hyperref package gives us a pdf with properly built
    % internal navigation ('pdf bookmarks' for the table of contents,
    % internal cross-reference links, web links for URLs, etc.)
    \usepackage{hyperref}
    \usepackage{longtable} % longtable support required by pandoc >1.10
    \usepackage{booktabs}  % table support for pandoc > 1.12.2
    

    
    
    \definecolor{orange}{cmyk}{0,0.4,0.8,0.2}
    \definecolor{darkorange}{rgb}{.71,0.21,0.01}
    \definecolor{darkgreen}{rgb}{.12,.54,.11}
    \definecolor{myteal}{rgb}{.26, .44, .56}
    \definecolor{gray}{gray}{0.45}
    \definecolor{lightgray}{gray}{.95}
    \definecolor{mediumgray}{gray}{.8}
    \definecolor{inputbackground}{rgb}{.95, .95, .85}
    \definecolor{outputbackground}{rgb}{.95, .95, .95}
    \definecolor{traceback}{rgb}{1, .95, .95}
    % ansi colors
    \definecolor{red}{rgb}{.6,0,0}
    \definecolor{green}{rgb}{0,.65,0}
    \definecolor{brown}{rgb}{0.6,0.6,0}
    \definecolor{blue}{rgb}{0,.145,.698}
    \definecolor{purple}{rgb}{.698,.145,.698}
    \definecolor{cyan}{rgb}{0,.698,.698}
    \definecolor{lightgray}{gray}{0.5}
    
    % bright ansi colors
    \definecolor{darkgray}{gray}{0.25}
    \definecolor{lightred}{rgb}{1.0,0.39,0.28}
    \definecolor{lightgreen}{rgb}{0.48,0.99,0.0}
    \definecolor{lightblue}{rgb}{0.53,0.81,0.92}
    \definecolor{lightpurple}{rgb}{0.87,0.63,0.87}
    \definecolor{lightcyan}{rgb}{0.5,1.0,0.83}
    
    % commands and environments needed by pandoc snippets
    % extracted from the output of `pandoc -s`
    \DefineVerbatimEnvironment{Highlighting}{Verbatim}{commandchars=\\\{\}}
    % Add ',fontsize=\small' for more characters per line
    \newenvironment{Shaded}{}{}
    \newcommand{\KeywordTok}[1]{\textcolor[rgb]{0.00,0.44,0.13}{\textbf{{#1}}}}
    \newcommand{\DataTypeTok}[1]{\textcolor[rgb]{0.56,0.13,0.00}{{#1}}}
    \newcommand{\DecValTok}[1]{\textcolor[rgb]{0.25,0.63,0.44}{{#1}}}
    \newcommand{\BaseNTok}[1]{\textcolor[rgb]{0.25,0.63,0.44}{{#1}}}
    \newcommand{\FloatTok}[1]{\textcolor[rgb]{0.25,0.63,0.44}{{#1}}}
    \newcommand{\CharTok}[1]{\textcolor[rgb]{0.25,0.44,0.63}{{#1}}}
    \newcommand{\StringTok}[1]{\textcolor[rgb]{0.25,0.44,0.63}{{#1}}}
    \newcommand{\CommentTok}[1]{\textcolor[rgb]{0.38,0.63,0.69}{\textit{{#1}}}}
    \newcommand{\OtherTok}[1]{\textcolor[rgb]{0.00,0.44,0.13}{{#1}}}
    \newcommand{\AlertTok}[1]{\textcolor[rgb]{1.00,0.00,0.00}{\textbf{{#1}}}}
    \newcommand{\FunctionTok}[1]{\textcolor[rgb]{0.02,0.16,0.49}{{#1}}}
    \newcommand{\RegionMarkerTok}[1]{{#1}}
    \newcommand{\ErrorTok}[1]{\textcolor[rgb]{1.00,0.00,0.00}{\textbf{{#1}}}}
    \newcommand{\NormalTok}[1]{{#1}}
    
    % Define a nice break command that doesn't care if a line doesn't already
    % exist.
    \def\br{\hspace*{\fill} \\* }
    % Math Jax compatability definitions
    \def\gt{>}
    \def\lt{<}
    % Document parameters
    \title{2015-03-26 Converting to paper form}
    
    
    

    % Pygments definitions
    
\makeatletter
\def\PY@reset{\let\PY@it=\relax \let\PY@bf=\relax%
    \let\PY@ul=\relax \let\PY@tc=\relax%
    \let\PY@bc=\relax \let\PY@ff=\relax}
\def\PY@tok#1{\csname PY@tok@#1\endcsname}
\def\PY@toks#1+{\ifx\relax#1\empty\else%
    \PY@tok{#1}\expandafter\PY@toks\fi}
\def\PY@do#1{\PY@bc{\PY@tc{\PY@ul{%
    \PY@it{\PY@bf{\PY@ff{#1}}}}}}}
\def\PY#1#2{\PY@reset\PY@toks#1+\relax+\PY@do{#2}}

\expandafter\def\csname PY@tok@gd\endcsname{\def\PY@tc##1{\textcolor[rgb]{0.63,0.00,0.00}{##1}}}
\expandafter\def\csname PY@tok@gu\endcsname{\let\PY@bf=\textbf\def\PY@tc##1{\textcolor[rgb]{0.50,0.00,0.50}{##1}}}
\expandafter\def\csname PY@tok@gt\endcsname{\def\PY@tc##1{\textcolor[rgb]{0.00,0.27,0.87}{##1}}}
\expandafter\def\csname PY@tok@gs\endcsname{\let\PY@bf=\textbf}
\expandafter\def\csname PY@tok@gr\endcsname{\def\PY@tc##1{\textcolor[rgb]{1.00,0.00,0.00}{##1}}}
\expandafter\def\csname PY@tok@cm\endcsname{\let\PY@it=\textit\def\PY@tc##1{\textcolor[rgb]{0.25,0.50,0.50}{##1}}}
\expandafter\def\csname PY@tok@vg\endcsname{\def\PY@tc##1{\textcolor[rgb]{0.10,0.09,0.49}{##1}}}
\expandafter\def\csname PY@tok@m\endcsname{\def\PY@tc##1{\textcolor[rgb]{0.40,0.40,0.40}{##1}}}
\expandafter\def\csname PY@tok@mh\endcsname{\def\PY@tc##1{\textcolor[rgb]{0.40,0.40,0.40}{##1}}}
\expandafter\def\csname PY@tok@go\endcsname{\def\PY@tc##1{\textcolor[rgb]{0.53,0.53,0.53}{##1}}}
\expandafter\def\csname PY@tok@ge\endcsname{\let\PY@it=\textit}
\expandafter\def\csname PY@tok@vc\endcsname{\def\PY@tc##1{\textcolor[rgb]{0.10,0.09,0.49}{##1}}}
\expandafter\def\csname PY@tok@il\endcsname{\def\PY@tc##1{\textcolor[rgb]{0.40,0.40,0.40}{##1}}}
\expandafter\def\csname PY@tok@cs\endcsname{\let\PY@it=\textit\def\PY@tc##1{\textcolor[rgb]{0.25,0.50,0.50}{##1}}}
\expandafter\def\csname PY@tok@cp\endcsname{\def\PY@tc##1{\textcolor[rgb]{0.74,0.48,0.00}{##1}}}
\expandafter\def\csname PY@tok@gi\endcsname{\def\PY@tc##1{\textcolor[rgb]{0.00,0.63,0.00}{##1}}}
\expandafter\def\csname PY@tok@gh\endcsname{\let\PY@bf=\textbf\def\PY@tc##1{\textcolor[rgb]{0.00,0.00,0.50}{##1}}}
\expandafter\def\csname PY@tok@ni\endcsname{\let\PY@bf=\textbf\def\PY@tc##1{\textcolor[rgb]{0.60,0.60,0.60}{##1}}}
\expandafter\def\csname PY@tok@nl\endcsname{\def\PY@tc##1{\textcolor[rgb]{0.63,0.63,0.00}{##1}}}
\expandafter\def\csname PY@tok@nn\endcsname{\let\PY@bf=\textbf\def\PY@tc##1{\textcolor[rgb]{0.00,0.00,1.00}{##1}}}
\expandafter\def\csname PY@tok@no\endcsname{\def\PY@tc##1{\textcolor[rgb]{0.53,0.00,0.00}{##1}}}
\expandafter\def\csname PY@tok@na\endcsname{\def\PY@tc##1{\textcolor[rgb]{0.49,0.56,0.16}{##1}}}
\expandafter\def\csname PY@tok@nb\endcsname{\def\PY@tc##1{\textcolor[rgb]{0.00,0.50,0.00}{##1}}}
\expandafter\def\csname PY@tok@nc\endcsname{\let\PY@bf=\textbf\def\PY@tc##1{\textcolor[rgb]{0.00,0.00,1.00}{##1}}}
\expandafter\def\csname PY@tok@nd\endcsname{\def\PY@tc##1{\textcolor[rgb]{0.67,0.13,1.00}{##1}}}
\expandafter\def\csname PY@tok@ne\endcsname{\let\PY@bf=\textbf\def\PY@tc##1{\textcolor[rgb]{0.82,0.25,0.23}{##1}}}
\expandafter\def\csname PY@tok@nf\endcsname{\def\PY@tc##1{\textcolor[rgb]{0.00,0.00,1.00}{##1}}}
\expandafter\def\csname PY@tok@si\endcsname{\let\PY@bf=\textbf\def\PY@tc##1{\textcolor[rgb]{0.73,0.40,0.53}{##1}}}
\expandafter\def\csname PY@tok@s2\endcsname{\def\PY@tc##1{\textcolor[rgb]{0.73,0.13,0.13}{##1}}}
\expandafter\def\csname PY@tok@vi\endcsname{\def\PY@tc##1{\textcolor[rgb]{0.10,0.09,0.49}{##1}}}
\expandafter\def\csname PY@tok@nt\endcsname{\let\PY@bf=\textbf\def\PY@tc##1{\textcolor[rgb]{0.00,0.50,0.00}{##1}}}
\expandafter\def\csname PY@tok@nv\endcsname{\def\PY@tc##1{\textcolor[rgb]{0.10,0.09,0.49}{##1}}}
\expandafter\def\csname PY@tok@s1\endcsname{\def\PY@tc##1{\textcolor[rgb]{0.73,0.13,0.13}{##1}}}
\expandafter\def\csname PY@tok@kd\endcsname{\let\PY@bf=\textbf\def\PY@tc##1{\textcolor[rgb]{0.00,0.50,0.00}{##1}}}
\expandafter\def\csname PY@tok@sh\endcsname{\def\PY@tc##1{\textcolor[rgb]{0.73,0.13,0.13}{##1}}}
\expandafter\def\csname PY@tok@sc\endcsname{\def\PY@tc##1{\textcolor[rgb]{0.73,0.13,0.13}{##1}}}
\expandafter\def\csname PY@tok@sx\endcsname{\def\PY@tc##1{\textcolor[rgb]{0.00,0.50,0.00}{##1}}}
\expandafter\def\csname PY@tok@bp\endcsname{\def\PY@tc##1{\textcolor[rgb]{0.00,0.50,0.00}{##1}}}
\expandafter\def\csname PY@tok@c1\endcsname{\let\PY@it=\textit\def\PY@tc##1{\textcolor[rgb]{0.25,0.50,0.50}{##1}}}
\expandafter\def\csname PY@tok@kc\endcsname{\let\PY@bf=\textbf\def\PY@tc##1{\textcolor[rgb]{0.00,0.50,0.00}{##1}}}
\expandafter\def\csname PY@tok@c\endcsname{\let\PY@it=\textit\def\PY@tc##1{\textcolor[rgb]{0.25,0.50,0.50}{##1}}}
\expandafter\def\csname PY@tok@mf\endcsname{\def\PY@tc##1{\textcolor[rgb]{0.40,0.40,0.40}{##1}}}
\expandafter\def\csname PY@tok@err\endcsname{\def\PY@bc##1{\setlength{\fboxsep}{0pt}\fcolorbox[rgb]{1.00,0.00,0.00}{1,1,1}{\strut ##1}}}
\expandafter\def\csname PY@tok@mb\endcsname{\def\PY@tc##1{\textcolor[rgb]{0.40,0.40,0.40}{##1}}}
\expandafter\def\csname PY@tok@ss\endcsname{\def\PY@tc##1{\textcolor[rgb]{0.10,0.09,0.49}{##1}}}
\expandafter\def\csname PY@tok@sr\endcsname{\def\PY@tc##1{\textcolor[rgb]{0.73,0.40,0.53}{##1}}}
\expandafter\def\csname PY@tok@mo\endcsname{\def\PY@tc##1{\textcolor[rgb]{0.40,0.40,0.40}{##1}}}
\expandafter\def\csname PY@tok@kn\endcsname{\let\PY@bf=\textbf\def\PY@tc##1{\textcolor[rgb]{0.00,0.50,0.00}{##1}}}
\expandafter\def\csname PY@tok@mi\endcsname{\def\PY@tc##1{\textcolor[rgb]{0.40,0.40,0.40}{##1}}}
\expandafter\def\csname PY@tok@gp\endcsname{\let\PY@bf=\textbf\def\PY@tc##1{\textcolor[rgb]{0.00,0.00,0.50}{##1}}}
\expandafter\def\csname PY@tok@o\endcsname{\def\PY@tc##1{\textcolor[rgb]{0.40,0.40,0.40}{##1}}}
\expandafter\def\csname PY@tok@kr\endcsname{\let\PY@bf=\textbf\def\PY@tc##1{\textcolor[rgb]{0.00,0.50,0.00}{##1}}}
\expandafter\def\csname PY@tok@s\endcsname{\def\PY@tc##1{\textcolor[rgb]{0.73,0.13,0.13}{##1}}}
\expandafter\def\csname PY@tok@kp\endcsname{\def\PY@tc##1{\textcolor[rgb]{0.00,0.50,0.00}{##1}}}
\expandafter\def\csname PY@tok@w\endcsname{\def\PY@tc##1{\textcolor[rgb]{0.73,0.73,0.73}{##1}}}
\expandafter\def\csname PY@tok@kt\endcsname{\def\PY@tc##1{\textcolor[rgb]{0.69,0.00,0.25}{##1}}}
\expandafter\def\csname PY@tok@ow\endcsname{\let\PY@bf=\textbf\def\PY@tc##1{\textcolor[rgb]{0.67,0.13,1.00}{##1}}}
\expandafter\def\csname PY@tok@sb\endcsname{\def\PY@tc##1{\textcolor[rgb]{0.73,0.13,0.13}{##1}}}
\expandafter\def\csname PY@tok@k\endcsname{\let\PY@bf=\textbf\def\PY@tc##1{\textcolor[rgb]{0.00,0.50,0.00}{##1}}}
\expandafter\def\csname PY@tok@se\endcsname{\let\PY@bf=\textbf\def\PY@tc##1{\textcolor[rgb]{0.73,0.40,0.13}{##1}}}
\expandafter\def\csname PY@tok@sd\endcsname{\let\PY@it=\textit\def\PY@tc##1{\textcolor[rgb]{0.73,0.13,0.13}{##1}}}

\def\PYZbs{\char`\\}
\def\PYZus{\char`\_}
\def\PYZob{\char`\{}
\def\PYZcb{\char`\}}
\def\PYZca{\char`\^}
\def\PYZam{\char`\&}
\def\PYZlt{\char`\<}
\def\PYZgt{\char`\>}
\def\PYZsh{\char`\#}
\def\PYZpc{\char`\%}
\def\PYZdl{\char`\$}
\def\PYZhy{\char`\-}
\def\PYZsq{\char`\'}
\def\PYZdq{\char`\"}
\def\PYZti{\char`\~}
% for compatibility with earlier versions
\def\PYZat{@}
\def\PYZlb{[}
\def\PYZrb{]}
\makeatother


    % Exact colors from NB
    \definecolor{incolor}{rgb}{0.0, 0.0, 0.5}
    \definecolor{outcolor}{rgb}{0.545, 0.0, 0.0}



    
    % Prevent overflowing lines due to hard-to-break entities
    \sloppy 
    % Setup hyperref package
    \hypersetup{
      breaklinks=true,  % so long urls are correctly broken across lines
      colorlinks=true,
      urlcolor=blue,
      linkcolor=darkorange,
      citecolor=darkgreen,
      }
    % Slightly bigger margins than the latex defaults
    
    \geometry{verbose,tmargin=1in,bmargin=1in,lmargin=1in,rmargin=1in}
    
    

    \begin{document}
    
    
    \maketitle
    
    

    
    \section{Updated translation and solution
method}\label{updated-translation-and-solution-method}

    Ian's most recent set of notes partitions the vector of variables $x$
into three parts:

\begin{itemize}
\item
  $x_1\in\mathbb{R}^{N_rT}$ contains wind deviations
\item
  $x_2\in\mathbb{R}^{(N+1)T}$ contains angle and mismatch variables
\item
  $x_3\in\mathbb{R}^T$ contains angle difference variables involved in
  line temperature calculation
\end{itemize}

With this notation, the problem becomes

\begin{align}
&& \min~ & x_1^\top Q_x x_1 \\
s.t. && Ax &= b \\
&& x_3^\top x_3 &= c
\end{align}

where $x=[x_1~x_2~x_3]^\top$.

\subsection{Translation}\label{translation}

As always, the first step in solving our problem is translation. We need
to go from $Ax=b$ to $Ay=0$, where $y=x-x^*$ and $x^*$ is the
translation. It is easy to find and translate by a point in the set
$\{x:Ax=b\}$, but we have an additional requirement: We don't want to
introduce any linear term to the norm constraint.

\begin{align}
&& \min~ & x_3^\top x_3 \\
s.t. && A_1x_1 + A_2x_2 + A_3x_3 &= b
\end{align}

    How do we partition $A$?

\begin{itemize}
\item
  $A_1$ contains columns of $A$ corresponding to wind deviations. The
  indices of these columns are indices for which $Q_{obj}$ is nonero.
\item
  $A_2$ contains columns corresponding to angles and mismatches. These
  are all the columns remaining after $A_1$ and $A_3$ are removed.
\item
  $A_3$ contains columns corresponding to angle difference variables.
  This is just the last $T$ columns of $A$.
\end{itemize}

    Once $A$ is partitioned, we can find $x_1^*$, $x_2^*$, and $x_3^*$. To
avoid messing up the norm, we need $x_3^*=0$. With that constraint
satisfied, we just need $A_1x_2 + A_2x_2 = b$, so we can obtain min-norm
$x_1^*$ and $x_2^*$ subject to this constraint.

    The translation itself is a change of variables from $x$ to
$y = x - x^*$. The translated problem is

\begin{align}
&& \min~ & y_1^\top Q_x y_1 + 2 y_1^\top x_1^* \\
s.t. && Ay &= 0 \\
&& y_3^\top y_3 &= c
\end{align}

    \subsection{Kernel mapping}\label{kernel-mapping}

Now that the problem has been translated, we replace $y$ by $Nz$, where
$N=[v_1~v_2~\ldots~v_k]\in\mathbb{R}^{n\times k}$ spans the
$k$-dimensional null space of $A$. This is somewhat similar to a
rotation, but it reduces the problem dimension from $n$ variables to $k$
variables.

From Ian's notes:

\begin{align}
\begin{bmatrix} y_1 \\ y_2 \\ y_3 \end{bmatrix}
&= \begin{bmatrix} N_1 \\ N_2 \\ N_3 \end{bmatrix} z
\end{align}

\begin{itemize}
\item
  The objective becomes
  $\min z^\top \hat{Q}_{obj} z + 2z^\top N_1x_1^*$, where
  $\hat{Q}_{obj} = N_1^\top Q N_1$
\item
  The temperature constraint becomes $z^\top N_3^\top N_3 z = c$.
\end{itemize}

    \subsection{Constraint Eigendecomposition and
rotation}\label{constraint-eigendecomposition-and-rotation}

After kernel mapping the constraint quadratic is no longer diagonal. We
can fix this by performing an Eigendecomposition
$N_3^\top N_3 = UDU^\top$ and letting $\hat{z} = U^\top z$ so that

\begin{align}
z^\top N_3^\top N_3 z &= \hat{z}^\top D\hat{z}
\end{align}

    $D$ has at most $T$ nonzero elements, because rank$(N _3) \leq T$ by
virtue of its dimension. It will look like this:

\begin{align}
D &= \begin{bmatrix} 0 & 0 \\ 0 & \hat{D} \end{bmatrix}
\end{align}

Now the constraint quadratic is diagonal, but we really need it to be a
norm constraint. We can effect this change by another change of
variables, this time from $\hat{z}$ to $w$.

Let $w = [w_1~w_2]^\top$ and $\hat{z}=[\hat{z}_1~\hat{z}_2]^\top$, and
relate them as follows:

\begin{align}
\begin{bmatrix} w_1 \\ w_2 \end{bmatrix} &=
\begin{bmatrix} I & 0 \\ 0 & \hat{D}^{1/2} \end{bmatrix}
\begin{bmatrix} \hat{z}_1 \\ \hat{z}_2 \end{bmatrix} \\
\implies w &= K\hat{z} = KU^\top z
\end{align}

    Then we can rewrite the constraint in terms of $w$:

\begin{align}
\hat{z}^\top D\hat{z} &= \hat{z}_2\hat{D}^{1/2}\hat{D}^{1/2}\hat{z}_2 \\
&= w_2^\top w_2
\end{align}

Note that $z = UK^{-1}w$ because $UU^\top = I$. Changing from $z$ to $w$
is equivalent to rotating by $(UK^{-1})^\top$.

Of course, this change of variables also influences the cost function:

\[ z^\top \hat{Q}_{obj}z = w^\top K^{-1}U^\top \hat{Q}_{obj}UK^{-1}w + 2z^\top K^{-1}U^\top N_1x_1^* = w^\top Bw + w^\top b\]

Thus, the optimization problem becomes

\begin{align}
&& \min~ w^\top Bw + w^\top b \\
s.t. && w_2^\top w_2 &= c
\end{align}

The purpose of this section was to change variables to obtain a norm
constraint. The next section eliminates $w_1$ using the KKT conditions
of the above problem. This will allow us to write the objective in terms
of $w_2$ only.

    \subsection{Eliminating $w_1$}\label{eliminating-wux5f1}

Note that $w_1$ is unconstrained. For a fixed $w_2$, we can use the KKT
conditions to find $w_1$ such that the objective is minimized. Begin by
expanding the objective:

\begin{align*} f(w_1,w_2) &=
\begin{bmatrix} w_1^\top & w_2^\top \end{bmatrix}
\begin{bmatrix} B_{11} & B_{12} \\ B_{12}^\top & B_{22}\end{bmatrix}
\begin{bmatrix} w_1 \\ w_2 \end{bmatrix} + 
\begin{bmatrix} w_1^\top & w_2^\top \end{bmatrix}
\begin{bmatrix} b_1 \\ b_2\end{bmatrix} \\
&=
w_1^\top B_{11}w_1 + 2w_1^\top B_{12}w_2 + w_2^\top B_{22}w_2 + w_1^\top b_1 + w_2^\top b_2
\end{align*}

Now set the partial derivative with respect to $w_1$ equal to zero:

\begin{align}
\nonumber \frac{\partial f}{\partial w_1} = 2w_1^\top B_{11} + 2w_2^\top B_{12}^\top + b_1^\top &= 0 \\
\nonumber \iff B_{11}w_1 + B_{12}w_2 + \frac{1}{2}b_1^\top &= 0 \\
\iff w_1 &= -B_{11}^{-1}\left(B_{12}w_2 - \frac{1}{2}b_1 \right)
\end{align}

After substituting this expression for $w_1$ into the objective and
simplifying, we obtain a new objective in terms of $w_2$ only:

\begin{align*}
f(w_2) &= w_2^\top\left(B_{22} - B_{12}^\top B_{11}^{-1} B_{12}\right)w_2 + 
w_2^\top (b_2 - B_{12}^\top B_{11}^{-1}b_1) - \frac{1}{4}b_1^\top B_{11}^{-1}b_1
\end{align*}

Note that the last term is constant and therefore plays no role in the
minimization. Thus, the optimization problem in terms of $w_2$ is:

\begin{align*}
&& \min~ w_2^\top \hat{B}w_2 + w_2^\top \hat{b} \\
s.t. && w_2^\top w_2 &= c \\ \nonumber\\
\text{where} && \hat{B} = B_{22} - B_{12}^\top B_{11}^{-1}B_{12} &\text{ and }\hat{b} = b_2 - B_{12}^\top B_{11}^{-1}b_1
\end{align*}

Now we are minimizing a quadratic objective subject to a norm
constraint. Note that $w_2$ has at most $T$ elements. Next we will
diagonalize the objective matrix $\hat{B}$.

    \subsection{Diagonalizing the objective
matrix}\label{diagonalizing-the-objective-matrix}

Is this necessary? $\hat{B}$ is already diagonal for the RTS-96\ldots{}

Let the Eigendecomposition of $\hat{B}$ be given by
$\hat{B} = \hat{U}\hat{D}\hat{U}^\top$ and perform a change of variables
to $\hat{w}_2 = \hat{U}^\top w_2$. Then the problem becomes

\begin{align*}
&& \min~ \hat{w}_2^\top \hat{D}\hat{w}_2 + \hat{w}_2^\top \hat{d} \\
s.t. && \hat{w}_2^\top \hat{w}_2 &= c
\end{align*}

Where $\hat{D}$ is a diagonal matrix and
$\hat{d} = \hat{U}^\top\hat{b}$. To solve this optimization problem, we
write the KKT conditions.

    \subsection{Solution via first-order optimality
conditions}\label{solution-via-first-order-optimality-conditions}

The gradient of the objective function is
$\nabla f(\hat{w_2}) =  2\hat{w}_2\hat{D} + \hat{d}$, and the gradient
of the constraint is $\nabla h(\hat{w}_2) = 2\hat{w}_2$. Letting $v$ be
the Lagrange multiplier associated with the constraint, we write

\begin{align}
\frac{\partial \mathcal{L}(\hat{w}_2,v)}{\partial \hat{w}_2} = 2\hat{D}\hat{w}_2 + \hat{d} - v(2\hat{w}_2) &= 0 \\
\iff \hat{D}\hat{w}_2 + \frac{1}{2}\hat{d} &= v\hat{w}_2
\end{align}

The above expression gives us $\hat{w}_2$ for any $v$ we choose. We seek
$v$ such that the corresponding $\hat{w}_2$ is also feasible (e.g.
$\hat{w}_2^\top \hat{w}_2 = c$).

Proceed according to Dan's notes: first, check $v=0$ and
$v=\hat{D}_{i,i}$ to rule these values out. Then write $\hat{w}_{2,i}$
in terms of $v$:

\begin{align}
\hat{w}_{2,i} &= \frac{1}{2}\left(\frac{\hat{d}_i}{v - \hat{D}_{i,i}}\right)
\end{align}

(Note that we have already ruled out values of $v$ that result in
vanishing denominators.) Substituting the above expression into the
constraint yields the secular equation:

\begin{align}
\frac{1}{4} \sum_{i}\left( \frac{\hat{d}_i}{v - \hat{D}_{i,i}}\right)^2 &= c
\end{align}

The secular equation has one pole per unique non-zero diagonal element
of $\hat{D}$. There are at most two solutions per pole: one to the left
and the other to the right. This is best understood graphically. Below
we have a secular equation with one pole. The horizontal axis is the
value of the Lagrange multiplier $v$, and the vertical axis is $s(v)$,
the value of the secular equation. The horizontal line is $s(v)=c$. The
two intersection points are the two solutions to this secular equation.

    


    % Add a bibliography block to the postdoc
    
    
    
    \end{document}
