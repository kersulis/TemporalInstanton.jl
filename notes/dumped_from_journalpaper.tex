% (In the notation of \eqref{eq:optobj}, $dev$ becomes $z_1$ and $\mathbf{Q}$ becomes $\mathbf{Q}_Z$.)

% TODO: generate a sensible precision matrix for the Polish network. Write ML best-fit code. Get some "typical" data from Yury that we can refer to. We just want something that makes sense. We are not inventing anything here.

% Don't delete what I do have. I can include it in my dissertation!

% TODO: assume no correlation; then we get X. Now use generated precision matrix and get Y. Interpret results briefly.

% Don't just say, "Hey it works!" Why would people care? What do we learn? Do a bunch of testing, but don't include it all. Just look for trends and interesting differences.

% TODO: get high-quality draft to Ian by next week.

% TODO: have a finished product by 2015-09-11. Ian can't do anything until he has a final draft.

% Submit to Transactions in 2 weeks.

% Connection with Jon: MPC to ensure line temps don't go too high. How to ensure feasibility? How do we know whether we have enough control effort to keep the temps down? He can use instanton analysis to find the minimal control effort needed to drive the temp back down. No correlation in this case; just want to establish the minimal amount of controllability required.

%There are many ways to obtain $\mathbf{Q}$ contains the same information as $\mathbf{C}$, but there are two advantages to the precision matrix perspective. First, it is possible to generate a precision matrix by fitting a small collection of parameters to observed data; it is not necessary to consider every pair of random variables. The following parameters, when combined with the wind site spatial layout, are sufficient to characterize a precision matrix:
%\begin{align*}
%p &= \left[ \kappa_1,\rho,\kappa_K,\sigma^2,a_{-1},b_0,b_{-1},b_1,c_0,c_{-1},c_1\right]^\top
%\end{align*}
%Second, \cite{tastu2015} showed that a good approximation to the precision matrix consists of tridiagonal blocks, implying that $\mathbf{Q}$ sparse. We refer the reader to \cite{tastu2015} for a detailed description of precision matrix generation. A brief overview of the steps will suffice for this paper.

%The first step is to encode the spatial layout of the network's wind farms so each site has at most four neighbors: North, South, West, and East. Figure \ref{fig:rts96adjacency_journal} illustrates spatial relationships for our modified RTS-96 network, whose eighteen wind sites are distributed across three uncorrelated areas.
%\begin{figure}[h]
%\centering
%\includegraphics[width=0.95\linewidth]{images/rts96adjacency_journal}
%\caption{Wind farm spatial relationships in a modified RTS-96 network. (Note: bus indices have been mapped to 1:73.)}
%\label{fig:rts96adjacency_journal}
%\end{figure}

%According to \cite{tastu2015}, one may generate a sufficiently accurate precision matrix by considering only adjacent neighbors and time steps for each site. Thus, for a site A at time $t$, one need only consider the behavior of its (at most four) neighbors at times $t-1$, $t$, and $t+1$; and behavior at the site itself for times $t-1$ and $t+1$. The requirement that $Q$ be positive definite further reduces the number of calculations one must perform. Ultimately, all calculations are based on the previously-described vector of parameters $p$. Maximum-likelihood estimation may be used to find a value for $p$ that fits observed wind data.

%
%% <REPLACE THIS SECTION BY THE PARAMETRIZED Q(\theta) SUMMARY>
%Begin by expressing $\mathbf{Q}$ as a product,
%\begin{align*}
%\mathbf{Q} &= \kappa \mathbf{B},
%\end{align*}
%where $\kappa$ is a diagonal coefficient matrix and $\mathbf{B}$ is a positive-definite standardized precision matrix. To obtain $\kappa$, repeat the diagonal matrix $\kappa_B$
%$N_R$ times:
%
%\begin{align*}
%\kappa &= \text{diag}(\kappa_B,\kappa_B,\ldots,\kappa_B),
%\end{align*}
%where $\kappa_B$ is a function of temporal boundaries $\kappa_1$ and $\kappa_T$; a ratio parameter $\rho$; and overall scaling $\sigma^2$:
%
%\begin{align*}
%\kappa_B = \begin{bmatrix} \kappa_1 & & & 0 \\
%& \kappa_2 & & \\
%& & \ddots & \\
%0 & & & \kappa_T \end{bmatrix}
%\end{align*}
%
%The first and last elements of $\kappa_B$ are fixed by temporal boundaries. Remaining values $\kappa_2$ to $\kappa_{T-1}$ increase with lead time according to the following model:
%\begin{align*}
%\kappa_i = \rho^{i-2},~~ i=2,\ldots,T-1
%\end{align*}
%Putting the pieces together, we obtain the following parametrization of $\kappa_B$ in terms of $\kappa_1$, $\kappa_k$, $\rho$, and $\sigma^2$:
%
%\begin{align*}
%\kappa_B = \frac{1}{\sigma^2} \begin{bmatrix}
%\kappa_1 & & & & & \\
%& 1 & & & 0 & \\
%& & \rho & & & \\
%& & & \ddots & & \\
%& 0 & & & \rho^{K-2} & \\
%& & & & & \kappa_K \end{bmatrix}
%\end{align*}
%
%
%The matrix $\mathbf{B}$ encodes spatial and temporal relationships between wind sites. We simplify spatial relationships so each wind site has at most four neighbors: North, South, West, and East. Figure \ref{fig:rts96adjacency_journal} illustrates spatial relationships for our modified RTS-96 network, whose eighteen wind sites are distributed across three uncorrelated areas.
%\begin{figure}[h]
%\centering
%\includegraphics[width=0.95\linewidth]{images/rts96adjacency_journal}
%\caption{Wind farm spatial relationships in a modified RTS-96 network. (Note: bus indices have been mapped to 1:73.)}
%\label{fig:rts96adjacency_journal}
%\end{figure}
%According to \cite{tastu2015}, it is sufficient to consider only adjacent neighbors and time steps. Thus, for a site A at time $t$, we need only consider the behavior of its four neighbors at times $t-1$, $t$, and $t+1$; and behavior at site A for times $t-1$ and $t+1$. The requirement that $\mathbf{B}$ be symmetric further reduces the number of elements we must generate.
%
%Together, $\kappa$ and $\mathbf{B}$ are based on eleven parameters.
%\begin{align*}
%p &= \left[ \kappa_1,\rho,\kappa_K,\sigma^2,a_{-1},b_0,b_{-1},b_1,c_0,c_{-1},c_1\right]^\top
%\end{align*}
%To assign numerical values to the vector of parameters $p$, we numerically optimize log-likelihood as described in \cite{tastu2015}.
%% </REPLACE THIS SECTION BY THE PARAMETRIZED Q(\theta) SUMMARY>
