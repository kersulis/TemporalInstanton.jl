\documentclass[10pt,letterpaper]{article}
\usepackage[latin1]{inputenc}
\usepackage{amsmath}
\usepackage{amsfonts}
\usepackage{amssymb}
\usepackage{graphicx}
\usepackage{framed}
\usepackage{pdfpages}

\author{Jonas Kersulis}
\title{Quals 2 Application Abstract}

\begin{document}
\maketitle
\subsubsection*{Instanton Analysis: Identifying Wind-induced Grid Vulnerability}

Wind fluctuations play risky games with today's electricity transmission networks. Because wind forecasts are often unreliable, transmission system operators cannot anticipate the wind's next move. Most fluctuations are harmless, but certain wind patterns could violate system constraints. Instanton analysis uses mathematical programming to find and rank these dangerous patterns according to their likelihood, giving system operators the information they need to prepare. Previous instanton analysis has provided promising results. This work focuses on two extensions. First, we replace the DC assumptions with approximate AC models to improve solution accuracy while maintaining problem convexity. In addition to enhancing instanton analysis, this work is a helpful resource to any grid practitioner comparing the various approximate AC models developed in recent years. Second, we consider multiple-time-step instanton analysis, where line constraints are expressed in terms of time-coupled thermal limits rather than instantaneous current limits. This formulation of the instanton problem more closely models the physical phenomenon of transmission line failure. Both instanton analysis extensions are illustrated on the IEEE RTS-96 network, and we conclude with a discussion of modeling challenges and future research directions.

\end{document}